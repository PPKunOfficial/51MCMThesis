\documentclass{51mcmthesis}

\title{论文题目}
\tihao{A}            % 题号
\duihao{4321}    % 报名号
\zubie{本科}
\schoolname{你的大学}
\membera{成员A}
\memberb{成员B}
\memberc{成员C}
\email{example@email.com}
\phone{1234567890}

\newcolumntype{Y}{>{\centering\arraybackslash}X}

\begin{document}
\maketitle
\keywords{\TeX{}\,  图片\,   表格\,  公式}
\begin{abstract}
    51mcmthesis 是为全国五一数学建模竞赛量身打造的 \LaTeX{} 模板,旨在帮助参赛队员更加专注于论文内容本身的撰写,而无需耗费大量时间在排版格式的调整和细节修饰上。模板充分考虑了比赛要求,整合了常用的结构和格式设置,并提供了若干定制化环境与命令,使得写作流程更加顺畅。使用本模板需要一定的 \LaTeX{} 基础,建议使用者熟悉常见宏包的基本用法,如文献引用、数学公式、图表插入、表格排版与列表环境等。模板附带示例文档 example.tex,可供参考学习和快速上手。

    \stextbf{本项目完全免费,遇到收费即为倒卖!另外使用需要一定\LaTeX{}基础,仅提供了封面和一些基本的格式设置,其他部分需要自行实现。本示例使用了过去的优秀论文作为示例,如有侵权,请联系作者删除。}

\end{abstract}
\section{问题重述}
标枪运动员如何取得最佳成绩依赖于运动员水平、标枪技术参数和比赛环境等
因素。为了探讨标枪出手时初始条件的良好组合规律,需要解决如下问题。

问题一:根据标枪示意图与某型标枪测量尺寸数据,估算该型标枪沿标枪中轴
线剖面面积、标枪表面积和标枪形心(剖面几何中心)的位置。

问题二:根据 24 名运动员标枪投掷的实测数据,建立数学模型模拟标枪在空
中的飞行轨迹,找到标枪飞行过程中的运动规律。

问题三:对标枪投掷出手瞬间、出手后受力及运动情况进行分析,建立标枪飞
行的数学模型。应用建立的模型估算某运动员(投掷出手速度为 29.70m/s、出手角
为 36.6°、初始攻角为-0.9°)所投掷标枪距离。再利用模型,给出运动员在出手
速度为 30 m/s 情况下使得投掷距离最大的出手角和初始攻角,并估算最大投掷距
离。

问题四:考虑风力的影响,建立数学模型。在出手速度为 31.7 m/s 的条件下,
应用模型,求解风向分别为顺风和逆风,风速分别为 3 m/s、6m/s 和 9m/s 条件
下,使得投掷距离最大的出手角、初始攻角和初始俯仰角速度,并估算最大投掷距
离。

问题五:建立数学模型分析出手速度、出手角、初始攻角、初始俯仰角速度、
风向及风速等要素对标枪投掷距离影响的相对重要性。
\section{问题假设}
\noindent\begin{enumerate}
    \item[假设1:] 顺风是从运动员的正后方吹正前方,逆风是从运动员正前方吹响正后方;
    \item[假设2:] 标枪的轮廓是连续光滑的曲线;
    \item[假设3:] 运动员的实际投掷可以达到理论上的初始条件;
    \item[假设4:] 标枪在飞行运动中做不发生形变;
    \item[假设5:] 不考虑温度、湿度等复杂因素对标枪性能的影响。
\end{enumerate}
\section{符号说明}
\begin{table}[H]
    \centering
    \begin{tabularx}{0.5\linewidth}{c Y}
        \toprule
        \textbf{符号} & \textbf{解释} \\
        \midrule
        $\upsilon$  & 投掷速度        \\
        \hline
        $\theta$    & 投掷角度        \\
        \hline
        $\alpha$    & 初始攻角        \\
        \hline
        $\beta$     & 初始俯仰角速度     \\
        \hline
        $\gamma$    & 风向          \\
        \hline
        $\delta$    & 风速          \\
        \bottomrule
    \end{tabularx}
\end{table}

\end{document}